% small.tex
\documentclass[xcolor=dvipsnames]{beamer}
\usepackage{lmodern}
\usepackage[T1]{fontenc}
\usepackage{graphicx}
\usepackage{ucs}
\usepackage[utf8x]{inputenc}
\usepackage[francais]{babel}
\setbeamertemplate{items}[ball]
\setbeamertemplate{blocks}[rounded][shadow=true]
\setbeamertemplate{navigation symbols}{}
\useoutertheme{infolines}
\author{Frédéric Lepied}
\title{AdminKit}
\usetheme{Madrid}
\usepackage{listings}

\begin{document}

\begin{frame}[plain]
  \titlepage
\end{frame}

\begin{frame}{Outline}
  \tableofcontents
\end{frame}

\section{Concepts}
\begin{frame}{Introduction}

What is AdminKit ?

\begin{itemize}
  \item a devops tool
  \item in the same vein as cfengine, puppet, chef...
  \item more in the spirit of cfengine V2 using the Python language
\end{itemize}

\end{frame}

\begin{frame}{Objectives}

What AdminKit do ?

\begin{itemize}
  \item configure and manage a set of Linux systems.
  \item maintain the configurations as defined.
  \item KISS principle.
\end{itemize}

What AdminKit do not do ?

\begin{itemize}
  \item manage the distribution of files $\rightarrow$ you need to use a way to
    distribute AdminKit files yourself: packages, version control
    system, web site, gui...
\end{itemize}

\end{frame}

\begin{frame}{Concepts}

\begin{itemize}
  \item One central configuration file describing the roles for each
    system you want to manage.
  \item Role files describe what is needed for fulfill the roles (packages,
    configuration files, services...).
  \item Configuration files are templates to allow easy customization
    and reuse.
\end{itemize}

\end{frame}

\section{Details}
\begin{frame}[fragile]{Central configuration}

\begin{itemize}
  \item Objective: define roles for all the systems and their
    associated variables.
  \item The central configuration file is stored under
    \tt{/var/lib/adminkit/adminkit.conf}
  \item It's a Python file so you can use conditionals, loops and functions...
  \item Example:
\lstset{language=Python,frame=single,numbers=left}
\begin{lstlisting}
define_domain('domain.com')
add_roles('host1', 'smtpserver', 'imapserver')
add_roles('host2', 'webserver')
add_var('host1', 'ip', '192.168.0.1')
add_var('host2', 'ip', '192.168.0.2')
\end{lstlisting}

\end{itemize}

\end{frame}

\begin{frame}{Central configuration functions}

\begin{description}
  \item[define\_domain(<name>)] define the domain name for the hosts
    used bellow this declaration.

  \item[add\_roles(<host>, <name1>, ...)] adds roles for \tt{<host>}.

  \item[add\_var(<host>, <name>, <value>)] define a variable for \tt{<host>}.

  \item[add\_to\_list(<host>, <name>, <value>)] add a <value> to the
    list \tt{<name>} for \tt{<host>}.

  \item[get\_var(<name>)] return the value of variable \tt{<name>}.
\end{description}

\end{frame}

\begin{frame}[fragile]{Role file}
  Role files are stored under \tt{/var/lib/adminkit/roles/}
\begin{itemize}
  \item Objectives: define the files, services and packages to manage
    for the role.
  \item Exemple:
\begin{lstlisting}
activate_service('ntp')
files_for_service('ntp', '/etc/ntp.conf')
\end{lstlisting}
\end{itemize}
\end{frame}

\begin{frame}{Role functions}

\begin{itemize}
  \item add\_files(<file desc>, <file desc> ...)

  \item files\_for\_service(<service>, <file1>, <file2>...)

  \item add\_dirs(<dir1>, <dir2> ...)

  \item check\_service\_by\_pidfile(<service>)

  \item activate\_service(<name>)
  
  \item deactivate\_service(<name>)
  
  \item check\_perms((<file>, <perm>), ...)

  \item add\_var(<name>[, <name2>...],<value>)

  \item add\_to\_list(<name>, <value>)

  \item get\_var(<name>)

  \item run\_once(<command>)

  \item files\_to\_command(<command>, <file1>, ...)

  \item install\_pkg(<pkg1>, <pkg2>...)

  \item global\_conf(<subdir>)
\end{itemize}

\end{frame}

\begin{frame}[fragile]{Config file}
  Config files are stored under \tt{/var/lib/adminkit/files/}
\begin{itemize}
  \item Objectives: file templates for easy reuse.
  \item jinja2 templates with access to all the variables defined by
    the roles and the global config file.
  \item the relative path under the root directory is the destination
    filename.
  \item Extract from \tt{/var/lib/adminkit/files/etc/ntp.conf} (will
    go to \tt{/etc/ntp.conf}):
\begin{lstlisting}

server {{ server }} iburst

\end{lstlisting}
\end{itemize}
\end{frame}

\begin{frame}[fragile]{Invoking}

Single command to know:

\begin{lstlisting}
# adminkit
\end{lstlisting}

Interesting options:

\begin{description}
  \item[-d] provide debug information.
  \item[-n] dry run.
  \item[-s] log to syslog.
\end{description}

The execution is done in 2 steps:
\begin{itemize}
  \item compute what needs to be done.
  \item execute changes to be able to do actions only once.
\end{itemize}
\end{frame}

\section{Advanced usages}

\begin{frame}[fragile]{Advanced usage: Global conf directive}

\begin{itemize}
  \item global\_conf is a role directive to be able to process the whole
  configuration of your park on a system.
  \item for example you want to generate Nagios files for all your servers automatically. You just have to declare this in your nagios role:
\begin{lstlisting}
global_conf('nagios')
\end{lstlisting}
        Then adminkit.conf will be read again by a special driver that
        will launch adminkit for all the servers setting their roles
        and using the root dir /var/lib/adminkit/nagios.
\end{itemize}
\end{frame}

\begin{frame}{Advanced usage: update-adminkit.conf}
\begin{itemize}
  \item Allow to split configuration in multiple files. By default
    lookup in /var/lib/adminkit/adminkit.conf.d/*.conf. Files are
    copied in order.
  \item Use the config file /etc/update-adminkit.conf to specify other
    directories to lookup configuration parts using the TOPDIRS shell
    variable. In these case, take files from
    \$DIR/\{*.conf,files,roles\}. Template files are copied, roles and
    conf files are concatenated in order from each directory.
\end{itemize}
\end{frame}

\begin{frame}{Thanks}
  \begin{center}
    \fbox{\textbf{Questions ?}}
  \end{center}
\end{frame}

\end{document}
